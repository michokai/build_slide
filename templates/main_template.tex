%----------------------------------------------------------------------------------------
%    PACKAGES AND THEMES
%----------------------------------------------------------------------------------------

\documentclass[aspectratio=169,xcolor=dvipsnames]{beamer}
% 2期のテーマ
\usetheme{SimpleDarkBlue}
% 3期のテーマ
%%%\usetheme{Madrid}  
%%%\usecolortheme{orchid} 
% \documentclass[aspectratio=43, 11pt, 11pt, t]{beamer}

%%%%%%%  metropolis-brown   %%%%%%%%%%
%% \documentclass[10pt]{beamer}
%% \usetheme[progressbar=frametitle]{metropolis}
%% \usepackage{appendixnumberbeamer}
%% \usepackage{metropolisbrown} % Don't forget to load this!
%%%%%%%  metropolis-brown   %%%%%%%%%%


\usepackage{luatexja}    % 日本語対応(LuaLaTeX専用)
%\usepackage{enumitem}
\usepackage{colortbl}
\usepackage{listings}
\usepackage[most]{tcolorbox}
\usepackage[table,xcdraw]{xcolor}
%\usepackage{listingsutf8}
\usepackage{array}  % ← これを preamble に追加
\usepackage{tabularx} % プリアンブルに追加(必要なら)
\usepackage{dirtree}
\usepackage{minted}
\usepackage{hyperref}
\usepackage{xcolor}
%\usepackage[colorlinks=true, linkcolor=blue, urlcolor=blue]{hyperref}
\usepackage{tikz}
\usepackage{graphicx} % Allows including images
\graphicspath{{images/}{../@@dir@@/images/}}
\usepackage{booktabs} % Allows the use of \toprule, \midrule and \bottomrule in tables
\usepackage{url} % プリアンブルに追加
\usepackage{makecell} 
\usepackage{lipsum} % サンプル用(不要なら削除)
\usepackage{setspace} % プリアンブルで
\usepackage{mathtools}
\usepackage{longtable}
\usepackage{pdfpages}
% 色・アイコン
\usepackage{xcolor}
\definecolor{ok}{HTML}{2E7D32}
\definecolor{warn}{HTML}{EF6C00}
\definecolor{info}{HTML}{1565C0}

\setbeamertemplate{footline}{
  \leavevmode%
  \hbox to \paperwidth{%
    \hfill
    \scriptsize \insertframenumber{} / \inserttotalframenumber%
    \hspace*{0.5cm}%
  }%
  \vspace{0.25cm}
}
%シンプルな青背景ボックスを定義して使う
\tcbset{
  mybluebox/.style={
    colback=cyan!10,      % 背景色(薄い水色)
    colframe=blue!50!black, % 枠の色
    boxrule=0.8pt,        % 枠線の太さ
    arc=1mm,              % 枠の角の丸み
    left=6pt, right=6pt, top=6pt, bottom=6pt,
    enhanced
  }
}

\lstdefinestyle{darkstyle}{
  backgroundcolor=\color{gray!20!black}, % 背景を濃いグレーに
  basicstyle=\ttfamily\color{white}\small, % 文字を白に
  frame=single,
  language=C
}
\lstdefinestyle{cmdstyle}{
  backgroundcolor=\color[RGB]{30,30,30}, % 黒に近い濃い灰色
  basicstyle=\ttfamily\color{yellow},    % 太字
  breaklines=true,
  frame=single,
  showstringspaces=false,
  rulecolor=\color{gray}
}

\lstset{
  language=Python,
  basicstyle=\ttfamily\small,
  frame=single,
  breaklines=true,
  xleftmargin=2em,
  escapechar=§,
%  escapeinside={(*@}{@*)}, % この中はLaTeXとして処理される
}

\definecolor{myDarkBlue}{RGB}{13,38,88}
\AtBeginSection[]{
  \begin{frame}[plain]
    \vfill
    \rule{\textwidth}{2pt}  % 上の線
    \begin{center}
      {\color{myDarkBlue}\Huge \textbf{\insertsection}}
    \end{center}
    \rule{\textwidth}{2pt}  % 下の線
    \vfill
  \end{frame}
}


%\mypausemodetrue % または \mypausemodefalse
% \newif\ifmypausemode
% \newcommand{\mypause}{%
%   \ifmypausemode
%     \pause
%   \fi
% }

% \def\techoutmode{0} % 1: 教師ノート非表示、0: 教師ノート表示
% 
% \newcommand{\teachernote}[1]{%
%   \ifnum\techoutmode=0
%     \begin{block}{教員メモ}
%     #1
%     \end{block}
%   \fi
% }
\usepackage{teacherframe}

% \newif\ifteachermode
% % \teachermodetrue % ← 教師用(表示)ならtrue、生徒用(非表示)ならfalse
% % \teachermodefalse % ← 生徒用(非表示)ならこっち
% \definecolor{myblue}{HTML}{E5F1F9}
% \definecolor{myred}{HTML}{B02418}
% \newcommand{\teacherframe}[2]{%
%   \ifteachermode{
%     % 一時的にタイトルバーの背景と文字色を変更
%     \setbeamercolor{frametitle}{bg=myblue}
%     \begin{frame}[allowframebreaks] % ← ここがポイント!
% 	 \frametitle{\textcolor{myred}{\textbf{#1}}}
%       #2
%     \end{frame}
% 	% もとに戻す(省略可:次のフレームで自動的にリセットされる)
% 	}
%   \fi
% }

\newcommand{\myfootertext}{@@footer@@}
\setbeamertemplate{footline}{
  \leavevmode
  \hbox to \paperwidth{
    \hspace*{0.2cm}
    \scriptsize\color{gray!50} \myfootertext
    % 右端ページ番号
    \hfill
    \scriptsize\color{gray} \insertframenumber{} / \inserttotalframenumber
    \hspace*{0.4cm}
  }
  \vspace{1pt}
}

\tcbuselibrary{listings, breakable}
% カスタム環境
\newtcolorbox{mycodebox}{
  colback=blue!5, colframe=blue!70!black,
  arc=1mm, boxrule=1pt, left=2mm, right=2mm, top=1mm, bottom=1mm,
  listing only, listing options={style=cmdstyle}
}

