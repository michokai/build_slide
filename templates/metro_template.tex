%----------------------------------------------------------------------------------------
%    PACKAGES AND THEMES metropolis
%----------------------------------------------------------------------------------------
\documentclass[aspectratio=169]{beamer}

% ---- Theme / Fonts ----
\usetheme{metropolis} % すっきり系。なければコメントアウト
\metroset{block=fill, sectionpage=progressbar, progressbar=foot}

% 日本語(LuaLaTeX推奨)
\usepackage{luatexja}
\usepackage{luatexja-fontspec}
% \setmainjfont{Noto Sans CJK JP} % 手元の日本語フォント名に合わせて変更可
% \setsansjfont{Noto Sans CJK JP}
%\setsansjfont{Hiragino Sans}
% 和文フォントの指定(macならヒラギノ角ゴ)
\setsansjfont{Hiragino Sans}[BoldFont={Hiragino Sans W6}]
% (必要なら明朝も:\setmainjfont{Hiragino Mincho ProN} など)
\setmonojfont{JetBrains Mono} % 等幅フォント(任意)
% 色・アイコン
\usepackage{xcolor}
\definecolor{ok}{HTML}{2E7D32}
\definecolor{warn}{HTML}{EF6C00}
\definecolor{info}{HTML}{1565C0}
\definecolor{bunrui}{HTML}{4EC2D8}

\usepackage[table]{xcolor}

\usepackage{graphicx} % Allows including images
\graphicspath{{images/}{../@@dir@@/images/}}
\usepackage[abs]{overpic}

% コードやコマンド
\usepackage{minted}  % コンパイル時に -shell-escape が必要
\setminted{
  fontsize=\footnotesize,
  breaklines=true,
  breaksymbolleft=\small\color{gray}{\ding{229}},
  frame=single
}
\usepackage{pifont}

\newcommand{\myfootertext}{@@footer@@}
\setbeamertemplate{footline}{
  \leavevmode
  \hbox to \paperwidth{
    \hspace*{0.2cm}
    \scriptsize\color{gray!50} \myfootertext
    % 右端ページ番号
    \hfill
    \scriptsize\color{gray} \insertframenumber{} / \inserttotalframenumber
    \hspace*{0.4cm}
  }
  \vspace{1pt}
}

%\mypausemodetrue % または \mypausemodefalse
% \newif\ifmypausemode
% \newcommand{\mypause}{%
%   \ifmypausemode
%     \pause
%   \fi
% }

\usepackage{teacherframe}

% \newif\ifteachermode
% \newcommand{\teacherframe}[2]{%
%   \ifteachermode{
%     % 一時的にタイトルバーの背景と文字色を変更
%     \setbeamercolor{frametitle}{bg=myblue}
%     \begin{frame}[allowframebreaks] % ← ここがポイント!
% 	 \frametitle{\textcolor{myred}{\textbf{#1}}}
%       #2
%     \end{frame}
% 	% もとに戻す(省略可:次のフレームで自動的にリセットされる)
% 	}
%   \fi
% }

\usepackage{array} % プリアンブルに追加

% C{幅} という列型を定義(中央寄せ+固定幅)
\newcolumntype{C}[1]{>{\centering\arraybackslash}p{#1}}
%\newcolumntype{M}[1]{>{\centering\arraybackslash}m{#1}} % middle の m
\newcolumntype{M}[1]{>{\raggedright\arraybackslash}m{#1}}




% 箇条書きの間隔や見た目
\setlength{\parskip}{4pt}

\usepackage{caption} % プリアンブルに追加
\usepackage{enumitem} % プリアンブルに追加
\usepackage{setspace} % プリアンブルに追加


\usepackage{dirtree}

% 黒丸を消して、テキストだけ表示する
\renewcommand*\DTstyle{\ttfamily}     % ノードのスタイル
\renewcommand*\DTstylecomment{\rmfamily\itshape\small} % コメント部分のスタイル